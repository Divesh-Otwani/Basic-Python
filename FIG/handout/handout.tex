\documentclass{article}

%%%
% Key commands: Spc m b to build
% Spc m v to view with default editor
%


%%%%%%%%%%%%%%%%%%%%%%%%%%%%%%%%%%%%%%%%%%%
% Here are packages for fonts, symbols, and graphics

\usepackage{lineno}

\usepackage{amsfonts}
\usepackage[english]{babel}
\usepackage{amssymb, latexsym, amsthm, amsmath, verbatim}
\usepackage[
                        top=0.8in, % applies to every page!
                        bottom=1in,
                        left=1.05in,
                        right=1.10in,
                        headheight=0in,
                        headsep=3in]{geometry}
                    %\addtolength{\textheight}{.2in}

                    
\usepackage{graphicx}
\graphicspath{{.}}
%\usepackage[linesnumbered,ruled]{algorithm2e}


%%%%%%%%%%%%%%%%%%%%%%%%%%%%%%%%%%%%%%%%%%%
% Here are useful environments. Use them by typing, e.g.
%  \begin{thm} Statement of theorem \end{thm}
% Often followed at some point by \begin{proof} Proof \end{proof}

\theoremstyle{theorem}

\newtheorem{thm}{Theorem}[section]
\newtheorem{lem}[thm]{\textbf Lemma}
\newtheorem{cor}[thm]{Corollary}
\newtheorem{prop}[thm]{\textbf Proposition}
\newtheorem{crit}[thm]{Criterium}
\newtheorem{alg}[thm]{Algorithm}

%%%%%%%%%%%%%%%%%%%%%%%%%%%%%%%%%%%%%%%%%
% Here is a different environment. Use it the same way and see what it
%looks like

\theoremstyle{definition}

\newtheorem{defn}[thm]{Definition}
\newtheorem{conj}[thm]{Conjecture}
\newtheorem{exmp}[thm]{\textbf{Examples}}
\newtheorem{exmps}[thm]{\textbf{Example}}
\newtheorem{prob}[thm]{Problem}

%%%%%%%%%%%%%%%%%%%%%%%%%%%%%%%%%%%%%%%%%
% To change how your theorems, exercises, etc. are numbered

\numberwithin{equation}{section}

%%%%%%%%%%%%%%%%%%%%%%%%%%%%%%%%%%%%%%%%%
% Here is a different environment. Use it the same way and see what it
%looks like
\theoremstyle{remark}

\newtheorem{rem}[thm]{\textbf{Remark}}
\newtheorem{note}[thm]{Note}
\newtheorem{claim}[thm]{Claim}
\renewcommand{\theclaim}{}
\newtheorem{summ}{Summary}
\renewcommand{\thesumm}{}
\newtheorem{case}{Case}
\newtheorem{ack}{ACKNOWLEDGEMENTS}
\renewcommand{\theack}{}

%%%%%%%%%%%%%%%%%%%%%%%%%%%%%%%%%%%%%%%
% Some macros for frequently used commands, and several ways to define them
\newcommand{\R}{\mathbb{R}}
\newcommand{\Q}{\mathbb{Q}}
\newcommand{\N}{\mathbb{N}}
\newcommand{\Z}{\mathbb{Z}}
\newcommand{\Ps}{\mathbb{P}}
\newcommand{\ba}{\backslash}
\DeclareMathOperator{\rank}{rank}

%This is a multi line comment
\newcommand{\comt}[1]{}
\newcommand{\spc}{\vspace{.06in}}
\newcommand{\f}[1]{$#1$}








\title{FIG := \text{FIG is good} \mid \text{FIG is great} \mid \text{FIG is god}
\\ ~ \\ \texttt{dotwani@haverford.edu | msoulanill@haverford.edu}  }
\date{}

\begin{document}
\maketitle
\vspace{-0.5in}

\section{What is FIG?}
Formally, FIG is a grammar of terms of infinite length defined through infinite
recursion.
But ... the club is something different.

\defn{FIG is a club that helps you build apps/programs to
  \begin{enumerate}
  \item
    Get internships
  \item
    Help Haverford (mainly the students)
  \item
    Realize your theory heavy education in a practical context
 }

\end{section}

\section{What do you do?}
\begin{enumerate}
\item You form a team of at most 3 people and find a project/app. (We can
  help you find a project if you can't think of a good one.)
\item
  Each week you and your teammates make some serious progress
  on building the app (with the goal of having it be done in 3-4 weeks, or, worst case,
  at the end of the semester). We'd like to make you put your code in a public
  repository in a FIG github organization\footnote{Look it up if you are unfamiliar.} so that way
  everyone's work is in one place (but you could not do that if you really want to).
\item
  One member of your team attends at least one of our bi-weekly half-an-hour
  meetings and takes 5 minutes to 
  \begin{enumerate}
  \item report what their team's progress was from last week
  \item give us a demo of their team's progress
  \item mention their team's timeline on finishing the app.
  \end{enumerate}

  After every team-representative finishes presenting, you can get help from us
  and each other on specific aspects of your project.
\item When you're done, you feel awesome about your project and what it does $-$ which often
  makes haverford way better.
\item
  You use your project on your resume (and at our recruiting event) to land internships.
\end{enumerate}

\end{section}

\newpage

\section{What do we do for you?}
\begin{itemize}
\item Keep you on track with your projects
\item Host weekly teaching sessions \textbf{with dinner} on things not covered in the haverford CS
  education like
  \begin{itemize}
  \item how to set up a development environment on your computer
  \item how to use git or some version control software, especially if you are
    on a team
  \item how to work on a project that takes a long time (like 3 months)
    and work well with your teammates
  \item how C or the basics of C++ works
  \item what pure functional languages are and how you can learn one
  \item how you can get good at learning new CS skills on your own (especially
    if your professor erroneously assumes you know something you don't) even
    with shitty documentation and masses of people online uttering pure nonsense
      
  \end{itemize}
\item Give you an awesome website of our projects, a list of our internship
  contacts, all sorts of internship and CS advice, lots of cool learning
  resources and no-bullshit discussions of important issues in CS.\footnote{This
    is coming shortly.}
\item Create a help email/slack on which you can post CS related questions
  -specific or broad.
\item Organize\footnote{Hopefully we can do this. It's not a sure thing yet; we
    need to talk to HIP and get this by some bureaucracy.} a end-of-semester recruiting event where
  representatives from companies watch you present your projects and offer you
  interviews or talk to you if they are interested.
\item Provide\footnote{Also not a sure thing yet.} a cool room with fast computers already set up for programming,
  useful books, nice chairs, CS decorations and maybe snacks. Teams can schedule
  times for this room.
  
  
\end{itemize}

%assets and how we help achieve those goals

\end{section}

\section{What should you do now?}
% think of project ideas, spread the word of FIG, find a group for some project
\begin{enumerate}
\item
  Spread the word of FIG. Tell your neighbors. Put up the flyers we're giving
  you.
\item
  Think of cool project ideas and if you want to, run them by us
\item Maybe find a team of at most 3 people interested in the same project as you.
  
\end{enumerate}

\end{document}
